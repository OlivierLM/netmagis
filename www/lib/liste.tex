\documentclass [a4paper,%ORIENTATION%] {article}

    % Package "francais" inadapt� car il met des espaces avant les ":"
    % ce qui est g�nant dans l'impression des adresses MAC et IPv6
    % \usepackage [francais] {babel}
    \usepackage [T1] {fontenc}

    \usepackage {palatino}
    \renewcommand {\familydefault} {phv}
    \usepackage {marvosym}			% pour l'euro (\EUR)

    \usepackage [pdftex,a4paper,margin=15mm,nohead,nofoot] {geometry}

    \usepackage {supertabular}

    \raggedbottom
    \frenchspacing
    \parskip=5mm
    \parindent=0mm

    % \renewcommand {\arraystretch} {1.4}                                             
\begin {document}

    \pdfinfo {
	/Title (liste.pdf)
	/Creator (DNS)
	/Producer (dns+arrgen+pdflatex)
	/Author (pda/jean)
	/Subject (Liste des machines)
	/Keywords (DNS, adresse IP)
    }

\begin {center}
    \Large\bf
    %% Universit� Louis Pasteur \\
    Centre R�seau Communication -- R�seau Osiris

    Liste des machines au %DATE%
\end {center}

% Pour avoir une ligne horizontale sur toute la largeur de la ligne
% \hrulefill

%NBMACHINES% machine%S% d�clar�e%S%.

%TABLEAU% 

\end{document}
