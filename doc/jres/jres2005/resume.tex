%%% si vous voulez compiler avec pdflatex
%\pdfoutput=1
\input {format.tex}
\documentclass {article}
    \usepackage {jres05-resume}

\begin {document}

%
% Le titre, les auteurs et les mots-clefs
%

\centerline {\LARGE{\bf {Le syst�me d'information du r�seau Osiris~:}}} \par
\vspace* {1mm}
\centerline {\LARGE{\bf {de la fibre optique jusqu'aux services}}}
\par
\vspace* {5mm}

\Auteur {Pierre David}
		{Centre R�seau Communication, Universit� Louis Pasteur}
		{Pierre.David} {crc.u-strasbg.fr}
\Auteur {Jean Benoit}
		{Centre R�seau Communication, Universit� Louis Pasteur}
		{Jean.Benoit} {crc.u-strasbg.fr}


\begin {motsclefs}
    Syst�me d'information r�seau, documentation, sch�mas r�seau,
    organisation
\end {motsclefs}

\begin {resume}
    Avec une complexit� technique croissante et une attente de
    fiabilit� de plus en plus forte des utilisateurs, les op�rateurs
    des r�seaux universitaires sont oblig�s d'industrialiser leur
    mode de gestion.

    Le CRC, gestionnaire du r�seau m�tropolitain strasbourgeois
    Osiris, s'est engag� dans une d�marche pour �laborer un syst�me
    d'information complet, avec comme objectifs de permettre la
    d�l�gation des op�rations, l'automatisation des t�ches
    d'exploitation, la documentation des informations et la visibilit�
    vers les correspondants r�seau.

    Partant de la base de l'application WebDNS, nous montrons comment
    notre approche s'est progressivement structur�e pour constituer
    � terme un ensemble complet et coh�rent. Int�grant un �l�ment
    souvent oubli�, les configurations des �quipements actifs, nous
    sommes � m�mes d'envisager des applications innovantes, comme
    par exemple la production automatique de sch�mas du r�seau ou
    la v�rification de coh�rence de certains points.

    Notre approche est tourn�e vers l'ensemble de la communaut�
    universitaire, puisque certains des outils sont d�j� diffus�s
    et les autres le seront si l'int�r�t le justifie.

\end {resume}

\end {document}
